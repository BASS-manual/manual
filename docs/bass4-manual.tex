\documentclass[]{book}
\usepackage{lmodern}
\usepackage{amssymb,amsmath}
\usepackage{ifxetex,ifluatex}
\usepackage{fixltx2e} % provides \textsubscript
\ifnum 0\ifxetex 1\fi\ifluatex 1\fi=0 % if pdftex
  \usepackage[T1]{fontenc}
  \usepackage[utf8]{inputenc}
\else % if luatex or xelatex
  \ifxetex
    \usepackage{mathspec}
  \else
    \usepackage{fontspec}
  \fi
  \defaultfontfeatures{Ligatures=TeX,Scale=MatchLowercase}
\fi
% use upquote if available, for straight quotes in verbatim environments
\IfFileExists{upquote.sty}{\usepackage{upquote}}{}
% use microtype if available
\IfFileExists{microtype.sty}{%
\usepackage{microtype}
\UseMicrotypeSet[protrusion]{basicmath} % disable protrusion for tt fonts
}{}
\usepackage{hyperref}
\hypersetup{unicode=true,
            pdftitle={BASS4 - ADMINISTRATOR'S MANUAL},
            pdfauthor={Louise Serenhov, Jenny-Li Örsell \& Brjánn Ljótsson},
            pdfborder={0 0 0},
            breaklinks=true}
\urlstyle{same}  % don't use monospace font for urls
\usepackage{natbib}
\bibliographystyle{apalike}
\usepackage{longtable,booktabs}
\usepackage{graphicx,grffile}
\makeatletter
\def\maxwidth{\ifdim\Gin@nat@width>\linewidth\linewidth\else\Gin@nat@width\fi}
\def\maxheight{\ifdim\Gin@nat@height>\textheight\textheight\else\Gin@nat@height\fi}
\makeatother
% Scale images if necessary, so that they will not overflow the page
% margins by default, and it is still possible to overwrite the defaults
% using explicit options in \includegraphics[width, height, ...]{}
\setkeys{Gin}{width=\maxwidth,height=\maxheight,keepaspectratio}
\IfFileExists{parskip.sty}{%
\usepackage{parskip}
}{% else
\setlength{\parindent}{0pt}
\setlength{\parskip}{6pt plus 2pt minus 1pt}
}
\setlength{\emergencystretch}{3em}  % prevent overfull lines
\providecommand{\tightlist}{%
  \setlength{\itemsep}{0pt}\setlength{\parskip}{0pt}}
\setcounter{secnumdepth}{5}
% Redefines (sub)paragraphs to behave more like sections
\ifx\paragraph\undefined\else
\let\oldparagraph\paragraph
\renewcommand{\paragraph}[1]{\oldparagraph{#1}\mbox{}}
\fi
\ifx\subparagraph\undefined\else
\let\oldsubparagraph\subparagraph
\renewcommand{\subparagraph}[1]{\oldsubparagraph{#1}\mbox{}}
\fi

%%% Use protect on footnotes to avoid problems with footnotes in titles
\let\rmarkdownfootnote\footnote%
\def\footnote{\protect\rmarkdownfootnote}

%%% Change title format to be more compact
\usepackage{titling}

% Create subtitle command for use in maketitle
\providecommand{\subtitle}[1]{
  \posttitle{
    \begin{center}\large#1\end{center}
    }
}

\setlength{\droptitle}{-2em}

  \title{BASS4 - ADMINISTRATOR'S MANUAL}
    \pretitle{\vspace{\droptitle}\centering\huge}
  \posttitle{\par}
    \author{Louise Serenhov, Jenny-Li Örsell \& Brjánn Ljótsson}
    \preauthor{\centering\large\emph}
  \postauthor{\par}
      \predate{\centering\large\emph}
  \postdate{\par}
    \date{Last updated 2019-07-29}

\usepackage{booktabs}
\usepackage{amsthm}
\makeatletter
\def\thm@space@setup{%
  \thm@preskip=8pt plus 2pt minus 4pt
  \thm@postskip=\thm@preskip
}
\makeatother

\begin{document}
\maketitle

{
\setcounter{tocdepth}{1}
\tableofcontents
}
\hypertarget{introduction}{%
\chapter{Introduction}\label{introduction}}

BASS is a flexible tool for creating online psychological treatment programs.
In this manual you will learn how to manage participants, combine self-help material into treatments, keep track on events during an ongoing study/program, manage security and privacy settings, collect and export data and communicate with participants through the administration interface of BASS.

\hypertarget{dictionary}{%
\chapter{DICTIONARY}\label{dictionary}}

These are recurrent concepts in the manual:

\textbf{Instrument (Formulär, mätning)}
An instrument is an electronic version of a paper form used during psychological assessment. Some examples of digitalized instruments are VAS (visual analogue scale), MADRS (Montgomery Åsberg Depression Rating Scale), SWLS (Satisfaction With Life Scale) and LSAS (Liebowitz Social Anxiety Scale).

\textbf{Assessment (Skattning)}
An assessment is a set of instruments, given in a specific order and at a specific occasion or for a specific number of occasions. A pre- and post-treatment assessment often consist of the same instruments with the afterward addition of one instrument measuring treatment satisfaction.

\textbf{Type}
A type represents the time-aspect of an assessment. Each assessment is linked to a type, typically SCREEN, PRE, POST or FOLLOW-UP or a customized type.

\textbf{Project}
A project is the administrative concept that connects a set of assessments to a set of participants.

\textbf{Participants}
A participant need to be assigned to a project to be able to fill in instruments and follow an assessment.

\textbf{Group}
A project can be divided into groups, and participants of the same group in a project can be managed collectively.

\hypertarget{login}{%
\chapter{LOGIN}\label{login}}

As soon as your database setup is ready, you can login to the administrator's interface. The interface is found at an URL of the format ``\url{https://webcbt.se/NameOfYourDatabase}''. Enter your credentials in the login box and press the Login button.

\hypertarget{the-main-menu}{%
\chapter{THE MAIN MENU}\label{the-main-menu}}

All functionality in the BASS administration interface can be accessed from the main menu to the left of your screen.

Which options are visible in the main menu depends on your authorization level. A usual setup is that one administrator manages the available instruments and assessments, while several therapists manage their own participants and individual treatments.

//Bild

\hypertarget{search-participants}{%
\chapter{SEARCH PARTICIPANTS}\label{search-participants}}

The ``Participant search'' is located at the top of the main menu. This is where you can search for and list participants by specific variables such as groups or projects.

//Bild

When you press ``Participant search'' you will see a view with four tabs:

//Bild

\begin{itemize}
\tightlist
\item
  Selection -- Add a filter to your search
\item
  Search -- Perform your search using text strings or other identifiers
\item
  Hidden columns -- View and show columns that are hidden
\item
  Save/load settings -- Save your recurrent searches for convenience
\end{itemize}

\hypertarget{selectionfilter}{%
\section{Selection/filter}\label{selectionfilter}}

If you press Selection, you can add a filter to your participant search.

//Bild

Here you can choose if you want to search all participants, your own participants (that you treat) or participants whose treatments you supervise.

You can also choose which project(s) or group(s) to search. The top two checkboxes can be used to quickly either mark or unmark all the below listed projects or groups. If no specific project or group is checked, all of them will be included in the search. This also means that unchecking all projects/groups won't return participants without a project/group.

\textbf{Hint:} If you want to search a specific group you should only mark that group, and not the corresponding project, as this will return all the participants belonging to the project and not only those in the group.

Your chosen selection of participants is shown in the participant list below the tab.

\hypertarget{search}{%
\section{Search}\label{search}}

The actual search is done in the \textbf{Search} tab. If you previously added search filters in the Selection tab they will now be active and delimit your search.

//Bild

Here you can use many different variables to search for one or several participants. The search is executed either automatically when you leave a filled-in search box or when you hit the Enter-key on your keyboard. Your search results are shown in the participant list below the tab.

Note that there is a discrepancy when searching by numbers or by text strings:

\begin{itemize}
\item
  Searching for the number ``12'' will only show the exact hit, while adding a \% sign to the search as in ``12\%'' will return both ``12'', ``123'' and ``012''.
\item
  Searching for the text string ``my'' will return both ``My'', ``Myra'' and ``Amy''. You don't need to add any \% sign for text string searches.
\end{itemize}

To search for several participants at the same time, you add a space between each corresponding search term in the search box.

\textbf{Hint:} This is useful if you want to search for participants whose IDs are listed on different rows in an Excel-file. Just copy the ID containing rows in Excel and directly paste them into the search box in BASS and they automatically receive a space between them.

\hypertarget{hide-show-and-sort-columns}{%
\section{Hide, show and sort columns}\label{hide-show-and-sort-columns}}

If you want to hide a column, you hover the mouse over the column header until a red X shows up. By pressing the X, the column will be hidden.

//Bild

To show/unhide a column, press the ``Hidden columns'' tab. This tab shows all hidden columns as buttons. Press the button with the column you want back and it will show up in the search results again.

//Bild

Most columns can be sorted alphabetically or by number. To sort a column, press the small up/down arrows that show when you hover over the column header.

\hypertarget{column-explanations}{%
\section{Column explanations}\label{column-explanations}}

There are a number of columns showing information, status or possible actions for a participant. Some are explained in the table below.

//Tabell

\hypertarget{saveload-search-settings}{%
\section{Save/load search settings}\label{saveload-search-settings}}

To save your current search settings, including both filters and search parameters, press the Save/load settings-tab. First ensure that the current search result for the settings you want to save are shown in the table below. Then write a name for your settings in the Currently loaded settings box and press ``Save as new''.

\textbf{Hint:} Be careful to not use the ``Save'' button instead, because this will overwrite any currently loaded settings including its name.

//Bild

The text ``saved!'' appears to the right of the buttons and your search is now saved and available in the dropdown below.

//Bild

The dropdown ``Saved settings'' is where you access all your previously saved search settings.

\textbf{Hint:} If you make a new search, the ``Currently loaded settings'' box may no longer reflect the content of the search result list below. To be sure that the list matches the settings you want to load, first select ``Choose'' in the dropdown menu and then reselect the settings you want.

\hypertarget{add-new-participant-to-group-and-change-group}{%
\subsection{5.5.1 ADD NEW PARTICIPANT TO GROUP AND CHANGE GROUP}\label{add-new-participant-to-group-and-change-group}}

It is possible to directly create a new participant within a specific project. This function is found below the table of participants. Just choose which project you want to add the new participant to, and you will be redirected to the ``New participant''-view with this project pre-filled.

//Bild

It is also easy to change which group a participant belongs to. For each participant in the table, you can choose a project in the dropdown in the Group column. Don't forget to save all changes by pressing the ``Save'' button below the table afterwards.

\hypertarget{assessments}{%
\chapter{Assessments}\label{assessments}}

Assessments are accessed from the ``Assessments'' option in the main menu. Note that you first have to choose a project in the dropdown in the main menu to make the Assessments option for that project visible. When you press ``Assessments'' you will see a view showing the existing assessments of the chosen project. All assessments that are listed in this view can be manually sorted with the upwards pointing arrow symbols to the right of each assessment name.

//Bild

You can show or hide the expanded overview by pressing the Show assessments overview-button. Here you can get a quick review of all the included assessments and their corresponding attributes.

\textbf{Hint:} Among other things, the assessment overview shows the order of each instrument in all assessments. This is a good place to ensure that the instrument order is kept from one assessment to another throughout the project. It also enables you to easily see if you somewhere have missed to include an instrument that should appear in several, similar assessments.

\hypertarget{create-or-edit-assessment}{%
\section{6.1 CREATE OR EDIT ASSESSMENT}\label{create-or-edit-assessment}}

Add a new assessment to your project by pressing ``Create new assessment'' at the bottom of the Assessment view. To instead edit an existing assessment, press the pencil symbol to the right of the name of the assessment you want to edit. This opens up the assessment panel where you can set a number of variables that define the assessment:

\hypertarget{name}{%
\subsection{NAME}\label{name}}

Here you can fill in a name for your assessment, for example \textbf{\emph{Screening}}.

\hypertarget{labelcustom-label}{%
\subsection{LABEL/CUSTOM LABEL}\label{labelcustom-label}}

You can either select one of the predefined labels in the drop-down, or write your own label in the Custom label textbox. Adding a custom label will surpass any predefined label that is selected from the drop-down. Note that the assessment label will be visible in reports when you export your data.

\textbf{Hint:} By selecting Weekly-assessment or Point-assessment some stats for Repetition (below) are preset.

\hypertarget{managed}{%
\subsection{MANAGED}\label{managed}}

This option sets whether data-gathering is managed individually or in groups.

\textbf{Hint:} If you have different cohorts, you may want to choose In group. Screening assessments are usually managed In group and these can be activated or deactivated for a certain group and date under Participants -\textgreater{} Groups -\textgreater{} screening group name -\textgreater{} Show -\textgreater{} Assessments.

\textbf{Hint:} If your participants start their treatments at different times, you usually choose Individually. The Individually option is also more flexible for long-term studies spanning over months when participants go for vacation and need some individual adjustment to the timing of assessments.

\hypertarget{repetition}{%
\subsection{REPETITION}\label{repetition}}

The Repetition option sets if the assessment is to be done once or repeatedly, and if so at what intervals and for how many times.

Assessments with the predefined label ``Weekly'' have repetition set to Weekly and the interval to 7 days.

Assessments with the label ``Point-assessment'' have repetition set to Manual. This means that the next assessment can be set manually to occur at an arbitrary date, independent of the time of the previous assessment. This is useful for assessments that are triggered by irregular events, for example a major flair of symptoms.

\hypertarget{time-limit}{%
\subsection{TIME LIMIT}\label{time-limit}}

Here you can set if participants have to fill out the assessment within a certain time limit.

\textbf{Important note:} Setting a time limit for an assessment is extremely important to prevent the results being mixed up with those from similar, subsequent assessments. For example, if an ongoing POST assessment is still accessible when the FOLLOW UP assessment is activated, the results of any of them is duplicated to the other. This results in data reports where no change seems to have occurred between the assessments.

An assessment with the time limit of 7 days that starts on a Monday will be available for the rest of that week but not for the next.

\textbf{Hint:} Keeping the time limit short, or shorter than the repetition interval, has the effect that participants fill in correct data corresponding to the set time-frame, but sometimes will miss the window when they can report. This is useful in assessments where accurate and time-dependent data is more important than full attendance.

\hypertarget{dependence}{%
\subsection{DEPENDENCE}\label{dependence}}

The Dependence option sets when the assessment is to be activated, in relation to the date of a previous assessment. The relationship is kept even if you change the date of the previous assessment.

Date offset from is where you select the previous assessment from which the date/delay is to be calculated.

\textbf{Note:} Setting Date offset from a reoccurring assessment (i.e.~WEEKLY) will count the delay from the date of the last assessment and not the first. If this is not what you want, consider creating a dummy assessment without instruments to hold the start/dependence date.

Checking Dynamic means that the delay is calculated from the time when the previous assessment was filled out instead of the time when it was scheduled. Note that this setting only can be done on individually managed assessments.

Delay is the number of days to wait before activation.

\textbf{Hint:} If you can't see the calculated date of your assessment in the view under Participants -\textgreater{} Groups -\textgreater{} group name -\textgreater{} Show -\textgreater{} Assessments, try to set the date of the previous interrelated assessment again and press the Save button.

\hypertarget{clinician-rated}{%
\subsection{CLINICIAN RATED}\label{clinician-rated}}

This option hides all instruments in the assessment for participants and instead enable clinicians to fill in the associated `clinician rated' instruments via the administration interface.

This setting allows a clinician to fill in the instrument(s) for a specific patient via Main menu -\textgreater{} Participants -\textgreater{} Groups -\textgreater{} specific group -\textgreater{} specific participant -\textgreater{} Assessments -\textgreater{} specific assessment -\textgreater{} specific instrument -\textgreater{} pen on document symbol

\textbf{Note:} Clinician rated instruments should not be added to self-assessments. Clinician-rated instruments are hidden for participants which makes it impossible for the participant to complete an assessment containing such an instrument.

\textbf{Hint:} Clinician rated assessments won't send automatic reminders. An option is to use flags instead to mark undone tasks.

\hypertarget{participants}{%
\chapter{Participants}\label{participants}}

\hypertarget{references}{%
\chapter{References}\label{references}}

\bibliography{bibliography.bib,packages.bib}


\end{document}
